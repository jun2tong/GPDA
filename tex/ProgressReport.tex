%%%%%%%%%%%%%%%%%%%%%%%%%%%%%%%%%%%%%%%%%%%%%%%%%%%%%%%%%%%%%%%%%%%%%%%%%%%%%%%%%%%%%%%%%%%%%%%%
%
% CSCI 1430 Project Progress Report Template
%
% This is a LaTeX document. LaTeX is a markup language for producing documents.
% Your task is to answer the questions by filling out this document, then to 
% compile this into a PDF document. 
% You will then upload this PDF to `Gradescope' - the grading system that we will use. 
% Instructions for upload will follow soon.
%
% 
% TO COMPILE:
% > pdflatex thisfile.tex
%
% If you do not have LaTeX and need a LaTeX distribution:
% - Departmental machines have one installed.
% - Personal laptops (all common OS): http://www.latex-project.org/get/
%
% If you need help with LaTeX, come to office hours. Or, there is plenty of help online:
% https://en.wikibooks.org/wiki/LaTeX
%
% Good luck!
% James and the 1430 staff
%
%%%%%%%%%%%%%%%%%%%%%%%%%%%%%%%%%%%%%%%%%%%%%%%%%%%%%%%%%%%%%%%%%%%%%%%%%%%%%%%%%%%%%%%%%%%%%%%%
%
% How to include two graphics on the same line:
% 
% \includegraphics[width=0.49\linewidth]{yourgraphic1.png}
% \includegraphics[width=0.49\linewidth]{yourgraphic2.png}
%
% How to include equations:
%
% \begin{equation}
% y = mx+c
% \end{equation}
% 
%%%%%%%%%%%%%%%%%%%%%%%%%%%%%%%%%%%%%%%%%%%%%%%%%%%%%%%%%%%%%%%%%%%%%%%%%%%%%%%%%%%%%%%%%%%%%%%%

\documentclass[11pt]{article}

\usepackage[english]{babel}
\usepackage[utf8]{inputenc}
\usepackage[colorlinks = true,
            linkcolor = blue,
            urlcolor  = blue]{hyperref}
\usepackage[a4paper,margin=1.5in]{geometry}
\usepackage{stackengine,graphicx}
\usepackage{fancyhdr}
\setlength{\headheight}{15pt}
\usepackage{microtype}
\usepackage{times}
\usepackage{booktabs}

% From https://ctan.org/pkg/matlab-prettifier
\usepackage[numbered,framed]{matlab-prettifier}

\frenchspacing
\setlength{\parindent}{0cm} % Default is 15pt.
\setlength{\parskip}{0.3cm plus1mm minus1mm}

\pagestyle{fancy}
\fancyhf{}
\lhead{Final Project Progress Report - 1}
\rhead{SYDE 671}
\rfoot{\thepage}

\date{}

\title{\vspace{-1cm}Final Project Progress Report - 1}


\begin{document}
\maketitle
\vspace{-2.25cm}
\thispagestyle{fancy}

\textbf{Team name: MaxMarginDB}

\textbf{Team members: JunYong Tong, Nick Torenvliet}

\textbf{TA name: Henry Leopold}

\section*{Project}
As a recap, \textbf{Unsupervised Visual Domain Adaptation: A Deep Max-Margin Approach} ~\cite{kim2019unsupervised} (MMDB) tackles the visual domain adaptation problem, where one learns from source-domain data and aims to perform well in target-domain data.
\begin{itemize}
    \item What is your project idea?
    \begin{itemize}
        \item We intend on training against two new datasets; datasets yet to be determined.
        \item We are investigating opportunities to improve MMDB performance by the use of alternate distributions in place of the current Gaussian Process.
        \item We are investigating opportunities to improve MMDB performance by the use of alternate optimization methods.
        \item We are investigating opportunities to improve MMDB training time by running it on multiple GPUs.
        \item We are investigating opportunities to improve MMDB performance by fine tuning hyper-parameters using a grid search. 
        \item We are investigating opportunities to map the MMDB process to a practical application.  
    \end{itemize}
    % \item What data have you collected?
    %     \begin{itemize}
    %     \item We are currently using SVHN (as source-domain) and MNIST (as target-domain) datasets.
    %     % \item We have collected the MNIST dataset http://yann.lecun.com/exdb/mnist/
    %     % \item We have collected the SVNH dataset http://ufldl.stanford.edu/housenumbers/
    %     \end{itemize}  
    % \item What software have you built or used?
    %     \begin{itemize}
    %     \item We have downloaded and run the software provided with the paper on local GPUs
    %     \item We are using Python v3.7 and PyTorch v1.2
    %     \end{itemize}  
    \item What has each team member contributed thus far?
        \begin{itemize}
        \item Jun and Nick have independently reviewed and assessed the MMDB method
        \item Jun has provided an excellent seminar on the derivation and mathematical basis of MMDB
        \item Jun and Nick have set up and run the code provided with MMDB
        \item Jun and Nick have have met for a brainstorming sessions
        \item Jun and Nick have established methods for remote collaboration over various channels provided by the internet
        \item Jun and Nick have provided input for the required project reporting.
        \end{itemize}  
    \item What intermediate results have you generated?
        \begin{itemize}
        \item We have successfully run the MMDB code on MNIST and SVNH and discussed performance of the out of the box run.
        \end{itemize}
    \item What problems have you faced or still have to consider?
    \begin{itemize}
        \item Default hyper-parameters provided by the authors do not yield the results they claimed in \cite{kim2019unsupervised}. We will be spending some time on finding the set of hyper-parameters that achieve results close to what they report.
        \item MNIST data was not available from the codebase. Seems to have different format as the one they use. As a result, we may not be able to exactly reproduce the results as claimed in the paper.
        \item PyTorch version improvement. Using PyTorch v1.2 now.
        \item MMDB is not immediately susceptible to parallelization. It may be that the best we can do is run parallel grid search. 
        \item Visualization of results is required.
    \end{itemize}
\end{itemize}

\bibliographystyle{ieee}
\bibliography{references}
\end{document}