%%%%%%%%%%%%%%%%%%%%%%%%%%%%%%%%%%%%%%%%%%%%%%%%%%%%%%%%%%%%%%%%%%%%%%%%%%%%%%%%%%%%%%%%%%%%%%%%
%
% CSCI 1430 Project Progress Report Template
%
% This is a LaTeX document. LaTeX is a markup language for producing documents.
% Your task is to answer the questions by filling out this document, then to 
% compile this into a PDF document. 
% You will then upload this PDF to `Gradescope' - the grading system that we will use. 
% Instructions for upload will follow soon.
%
% 
% TO COMPILE:
% > pdflatex thisfile.tex
%
% If you do not have LaTeX and need a LaTeX distribution:
% - Departmental machines have one installed.
% - Personal laptops (all common OS): http://www.latex-project.org/get/
%
% If you need help with LaTeX, come to office hours. Or, there is plenty of help online:
% https://en.wikibooks.org/wiki/LaTeX
%
% Good luck!
% James and the 1430 staff
%
%%%%%%%%%%%%%%%%%%%%%%%%%%%%%%%%%%%%%%%%%%%%%%%%%%%%%%%%%%%%%%%%%%%%%%%%%%%%%%%%%%%%%%%%%%%%%%%%
%
% How to include two graphics on the same line:
% 
% \includegraphics[width=0.49\linewidth]{yourgraphic1.png}
% \includegraphics[width=0.49\linewidth]{yourgraphic2.png}
%
% How to include equations:
%
% \begin{equation}
% y = mx+c
% \end{equation}
% 
%%%%%%%%%%%%%%%%%%%%%%%%%%%%%%%%%%%%%%%%%%%%%%%%%%%%%%%%%%%%%%%%%%%%%%%%%%%%%%%%%%%%%%%%%%%%%%%%

\documentclass[11pt]{article}

\usepackage[english]{babel}
\usepackage[utf8]{inputenc}
\usepackage[colorlinks = true,
            linkcolor = blue,
            urlcolor  = blue]{hyperref}
\usepackage[a4paper,margin=1.5in]{geometry}
\usepackage{stackengine,graphicx}
\usepackage{fancyhdr}
\setlength{\headheight}{15pt}
\usepackage{microtype}
\usepackage{times}
\usepackage{booktabs}

% From https://ctan.org/pkg/matlab-prettifier
\usepackage[numbered,framed]{matlab-prettifier}

\frenchspacing
\setlength{\parindent}{0cm} % Default is 15pt.
\setlength{\parskip}{0.3cm plus1mm minus1mm}

\pagestyle{fancy}
\fancyhf{}
\lhead{Final Project Progress Report - 2}
\rhead{SYDE 671}
\rfoot{\thepage}

\date{}

\title{\vspace{-1cm}Final Project Progress Report - 2}


\begin{document}
\maketitle
\vspace{-2.25cm}
\thispagestyle{fancy}

\textbf{Team name: MaxMarginDB}

\textbf{Team members: JunYong Tong, Nick Torenvliet}

\textbf{TA name: Henry Leopold}

\section*{Project}
\begin{itemize}
  \item What is your project idea?
  \begin{itemize}
      \item Overall idea did not change still replicating results from \textbf{Unsupervised Visual Domain Adaptation: A Deep Max-Margin Approach} (MMDB) and improve upon their methodology.
      \item Attempting grid search for result replication.
      \item Investigating the use of natural gradient for variational inference optimization.
      \item Investigating the algorithm performance if we are to reverse the source and target.
  \end{itemize}
  \item What data have you collected?
  \begin{itemize}
      \item We are using SVHN and MNIST data.
  \end{itemize}
  \item What software have you built or used?
  \begin{itemize}
      \item Got the source code running in both Nick's and Jun's system.
  \end{itemize}
  \item What has each team member contributed thus far?
  \begin{itemize}
      \item Nick coded up batch job for grid search.
      \item Nick contributed ideas for practical applications.
      \item Jun analysed the methodology step-by-step and identified a feasible opportunity for improvement.
  \end{itemize}
  \item What intermediate results have you generated?
  \begin{itemize}
      \item Toyed around with the code and identified key parameters for grid search.
  \end{itemize}
  \item What problems have you faced or still have to consider?
  \begin{itemize}
      \item Each experiment for a set of parameters takes around a day to complete, so given the time frame, we can only replicate the key results provided in the paper. We will not be able to replicate all experiments in the paper.
      \item Setting up batch job with Pytorch across multiple GPUs is difficult due to default multiprocessing settings in PyTorch.
  \end{itemize}
\end{itemize}

% \bibliographystyle{ieee}
% \bibliography{references}
\end{document}